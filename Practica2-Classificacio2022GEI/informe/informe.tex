\documentclass{article}
\usepackage[utf8]{inputenc}
\usepackage[catalan]{babel}
\usepackage{amssymb}        % símbols de l'AMS
\usepackage{amsmath}        % macros de l'AMS
\usepackage[pdftex]{graphicx}  % poder incloure gráfics
\usepackage[pdftex]{color}     % poder fer servir color al text
\usepackage{multicol}
\usepackage{amsthm}
\usepackage{listings}
\usepackage{hyperref}
\usepackage{geometry}
\usepackage{tikz}
\usepackage{pgfplots}
\usepackage{subcaption}
\usepackage[catalan]{babel}
\usepackage{mathtools}
\usepackage{geometry}
\usepackage{amsmath}
\usepackage{amssymb}
\usepackage{fancyhdr}
\usepackage{multirow}
%\usepackage[table,xcdraw]{xcolor}
\usepackage{float}
\usepackage{verbatim}
\usepackage{pgfplots}
\pgfplotsset{compat=newest}
\usepgfplotslibrary{fillbetween}
\usetikzlibrary{patterns}
\usepackage{vmargin} 				%margenes



\title{Pràctica 2: Classificació}
\author{Guillermo Vivancos Alonso 1606206\\
	Javier Esmoris Cerezuela 1498396\\
	Oriol Marión Escudé 1566740}
\begin{document}
	\date{}
	\setpapersize{A4}
	\setmargins{2.5cm}       % margen izquierdo
	{1.5cm}                        % margen superior
	{16.5cm}                      % anchura del texto
	{23.42cm}                    % altura del texto
	{10pt}                           % altura de los encabezados
	{1cm}                           % espacio entre el texto y los encabezados
	{0pt}                             % altura del pie de página
	{2cm}                           % espacio entre el texto y el pie de página
	
	\pagestyle{fancy}
	\fancyhf{}
	\lhead{\textbf{Aprenentatge Computacional 102787} \newline Pràctica 2: Classificació\\}
	\rhead{\hfill \textbf{
			Guillermo Vivancos Alonso 1606206\\
			Javier Esmoris Cerezuela 1498396\\
			Oriol Marión Escudé 1566740}}
	\rfoot{\thepage}
	\maketitle
	\noindent
	\section*{Introducció}
	En aquesta pràctica analitzarem una base de dades sobre la potabilitat de l'aigua i les diferents concentracions d'algunes substàncies. Veurem quines distribucions tenen els atributs, la relació entre ells i intentarem determinar si donada una mostra, aquesta és potable o no.
	
	\section*{Apartat B}
	Els atributs que tenim a la base de dades són els següents:
	\begin{enumerate}
		\addtocounter{enumi}{-1}
		\item pH [float]: pH de l'aigua.
		\item Hardness [float]:concentració de calci i magnesi.
		\item Solids [float]: edat del jugador mesurada en anys i dies.
		\item Chloramines [float]: concentració de cloramines. 
		\item Sulfate [float]: concentració de sulfats.
		\item Conductivity [float]: conductivitat de l'aigua.
		\item Organic\_carbon [float]: concentració de compostos orgànics.
		\item Trihalomethanes [float]: concentració de trihalometans.
		\item Turbidity [float]: Terbolesa de l'aigua.
		\item Potability [integer]: 0 si no és potable, 1 si és potable.
	\end{enumerate}
	Tenim 10 atributs a la base de dades, tots de tipus float menys l'últim que és binari, per tant l'atribut categòric només pot prendre dos valors.\\
	\\
	Pel que fa a les correlacions amb l'atribut categòric, totes són poc significatives com es pot veure en la figura~\ref{fig:correlacions}.\\
	\begin{figure}[!h]
		\centering
		\includegraphics[width=0.4\linewidth]{../images/correlacions}
		\caption{Correlació entre els atributs}
		\label{fig:correlacions}
	\end{figure}\\
	A continuació veurem la distribució de la potabilitat~\ref{fig:distribuciopotabilitat} i les distribucions dels altres atributs segons la potabilitat~\ref{fig:distribucions}.\\
	\\
	\begin{figure}[!h]
		\centering
		\includegraphics[width=0.4\linewidth]{../images/distribucio_potabilitat}
		\caption{Distribucio de la potabilitat}
		\label{fig:distribuciopotabilitat}
	\end{figure}
	\begin{figure}[!h]
		\centering
		\includegraphics[width=0.7\linewidth]{../images/distribucions}
		\caption{Distribucions dels atributs segons la potabilitat}
		\label{fig:distribucions}
	\end{figure}
	A simple vista podem veure en la figura~\ref{fig:distribucions} que
	\newpage
	\section*{Apartat A}

	
	
\end{document}